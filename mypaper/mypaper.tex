
%% bare_conf_compsoc.tex
%% V1.4b
%% 2015/08/26
%% by Michael Shell
%% See:
%% http://www.michaelshell.org/
%% for current contact information.
%%
%% This is a skeleton file demonstrating the use of IEEEtran.cls
%% (requires IEEEtran.cls version 1.8b or later) with an IEEE Computer
%% Society conference paper.
%%
%% Support sites:
%% http://www.michaelshell.org/tex/ieeetran/
%% http://www.ctan.org/pkg/ieeetran
%% and
%% http://www.ieee.org/

%%*************************************************************************
%% Legal Notice:
%% This code is offered as-is without any warranty either expressed or
%% implied; without even the implied warranty of MERCHANTABILITY or
%% FITNESS FOR A PARTICULAR PURPOSE!
%% User assumes all risk.
%% In no event shall the IEEE or any contributor to this code be liable for
%% any damages or losses, including, but not limited to, incidental,
%% consequential, or any other damages, resulting from the use or misuse
%% of any information contained here.
%%
%% All comments are the opinions of their respective authors and are not
%% necessarily endorsed by the IEEE.
%%
%% This work is distributed under the LaTeX Project Public License (LPPL)
%% ( http://www.latex-project.org/ ) version 1.3, and may be freely used,
%% distributed and modified. A copy of the LPPL, version 1.3, is included
%% in the base LaTeX documentation of all distributions of LaTeX released
%% 2003/12/01 or later.
%% Retain all contribution notices and credits.
%% ** Modified files should be clearly indicated as such, including  **
%% ** renaming them and changing author support contact information. **
%%*************************************************************************


% *** Authors should verify (and, if needed, correct) their LaTeX system  ***
% *** with the testflow diagnostic prior to trusting their LaTeX platform ***
% *** with production work. The IEEE's font choices and paper sizes can   ***
% *** trigger bugs that do not appear when using other class files.       ***                          ***
% The testflow support page is at:
% http://www.michaelshell.org/tex/testflow/



\documentclass[conference,compsoc]{IEEEtran}
% Some/most Computer Society conferences require the compsoc mode option,
% but others may want the standard conference format.
%
% If IEEEtran.cls has not been installed into the LaTeX system files,
% manually specify the path to it like:
% \documentclass[conference,compsoc]{../sty/IEEEtran}





% Some very useful LaTeX packages include:
% (uncomment the ones you want to load)


% *** MISC UTILITY PACKAGES ***
%
%\usepackage{ifpdf}
% Heiko Oberdiek's ifpdf.sty is very useful if you need conditional
% compilation based on whether the output is pdf or dvi.
% usage:
% \ifpdf
%   % pdf code
% \else
%   % dvi code
% \fi
% The latest version of ifpdf.sty can be obtained from:
% http://www.ctan.org/pkg/ifpdf
% Also, note that IEEEtran.cls V1.7 and later provides a builtin
% \ifCLASSINFOpdf conditional that works the same way.
% When switching from latex to pdflatex and vice-versa, the compiler may
% have to be run twice to clear warning/error messages.






% *** CITATION PACKAGES ***
%
\ifCLASSOPTIONcompsoc
  % IEEE Computer Society needs nocompress option
  % requires cite.sty v4.0 or later (November 2003)
  \usepackage[nocompress]{cite}
\else
  % normal IEEE
  \usepackage{cite}
\fi

\usepackage{verbatim}
\usepackage{amsmath}
% cite.sty was written by Donald Arseneau
% V1.6 and later of IEEEtran pre-defines the format of the cite.sty package
% \cite{} output to follow that of the IEEE. Loading the cite package will
% result in citation numbers being automatically sorted and properly
% "compressed/ranged". e.g., [1], [9], [2], [7], [5], [6] without using
% cite.sty will become [1], [2], [5]--[7], [9] using cite.sty. cite.sty's
% \cite will automatically add leading space, if needed. Use cite.sty's
% noadjust option (cite.sty V3.8 and later) if you want to turn this off
% such as if a citation ever needs to be enclosed in parenthesis.
% cite.sty is already installed on most LaTeX systems. Be sure and use
% version 5.0 (2009-03-20) and later if using hyperref.sty.
% The latest version can be obtained at:
% http://www.ctan.org/pkg/cite
% The documentation is contained in the cite.sty file itself.
%
% Note that some packages require special options to format as the Computer
% Society requires. In particular, Computer Society  papers do not use
% compressed citation ranges as is done in typical IEEE papers
% (e.g., [1]-[4]). Instead, they list every citation separately in order
% (e.g., [1], [2], [3], [4]). To get the latter we need to load the cite
% package with the nocompress option which is supported by cite.sty v4.0
% and later.





% *** GRAPHICS RELATED PACKAGES ***
%
\ifCLASSINFOpdf
  % \usepackage[pdftex]{graphicx}
  % declare the path(s) where your graphic files are
  % \graphicspath{{../pdf/}{../jpeg/}}
  % and their extensions so you won't have to specify these with
  % every instance of \includegraphics
  % \DeclareGraphicsExtensions{.pdf,.jpeg,.png}
\else
  % or other class option (dvipsone, dvipdf, if not using dvips). graphicx
  % will default to the driver specified in the system graphics.cfg if no
  % driver is specified.
  % \usepackage[dvips]{graphicx}
  % declare the path(s) where your graphic files are
  % \graphicspath{{../eps/}}
  % and their extensions so you won't have to specify these with
  % every instance of \includegraphics
  % \DeclareGraphicsExtensions{.eps}
\fi
% graphicx was written by David Carlisle and Sebastian Rahtz. It is
% required if you want graphics, photos, etc. graphicx.sty is already
% installed on most LaTeX systems. The latest version and documentation
% can be obtained at:
% http://www.ctan.org/pkg/graphicx
% Another good source of documentation is "Using Imported Graphics in
% LaTeX2e" by Keith Reckdahl which can be found at:
% http://www.ctan.org/pkg/epslatex
%
% latex, and pdflatex in dvi mode, support graphics in encapsulated
% postscript (.eps) format. pdflatex in pdf mode supports graphics
% in .pdf, .jpeg, .png and .mps (metapost) formats. Users should ensure
% that all non-photo figures use a vector format (.eps, .pdf, .mps) and
% not a bitmapped formats (.jpeg, .png). The IEEE frowns on bitmapped formats
% which can result in "jaggedy"/blurry rendering of lines and letters as
% well as large increases in file sizes.
%
% You can find documentation about the pdfTeX application at:
% http://www.tug.org/applications/pdftex





% *** MATH PACKAGES ***
%
%\usepackage{amsmath}
% A popular package from the American Mathematical Society that provides
% many useful and powerful commands for dealing with mathematics.
%
% Note that the amsmath package sets \interdisplaylinepenalty to 10000
% thus preventing page breaks from occurring within multiline equations. Use:
%\interdisplaylinepenalty=2500
% after loading amsmath to restore such page breaks as IEEEtran.cls normally
% does. amsmath.sty is already installed on most LaTeX systems. The latest
% version and documentation can be obtained at:
% http://www.ctan.org/pkg/amsmath





% *** SPECIALIZED LIST PACKAGES ***
%
%\usepackage{algorithmic}
% algorithmic.sty was written by Peter Williams and Rogerio Brito.
% This package provides an algorithmic environment fo describing algorithms.
% You can use the algorithmic environment in-text or within a figure
% environment to provide for a floating algorithm. Do NOT use the algorithm
% floating environment provided by algorithm.sty (by the same authors) or
% algorithm2e.sty (by Christophe Fiorio) as the IEEE does not use dedicated
% algorithm float types and packages that provide these will not provide
% correct IEEE style captions. The latest version and documentation of
% algorithmic.sty can be obtained at:
% http://www.ctan.org/pkg/algorithms
% Also of interest may be the (relatively newer and more customizable)
% algorithmicx.sty package by Szasz Janos:
% http://www.ctan.org/pkg/algorithmicx




% *** ALIGNMENT PACKAGES ***
%
%\usepackage{array}
% Frank Mittelbach's and David Carlisle's array.sty patches and improves
% the standard LaTeX2e array and tabular environments to provide better
% appearance and additional user controls. As the default LaTeX2e table
% generation code is lacking to the point of almost being broken with
% respect to the quality of the end results, all users are strongly
% advised to use an enhanced (at the very least that provided by array.sty)
% set of table tools. array.sty is already installed on most systems. The
% latest version and documentation can be obtained at:
% http://www.ctan.org/pkg/array


% IEEEtran contains the IEEEeqnarray family of commands that can be used to
% generate multiline equations as well as matrices, tables, etc., of high
% quality.




% *** SUBFIGURE PACKAGES ***
%\ifCLASSOPTIONcompsoc
%  \usepackage[caption=false,font=footnotesize,labelfont=sf,textfont=sf]{subfig}
%\else
%  \usepackage[caption=false,font=footnotesize]{subfig}
%\fi
% subfig.sty, written by Steven Douglas Cochran, is the modern replacement
% for subfigure.sty, the latter of which is no longer maintained and is
% incompatible with some LaTeX packages including fixltx2e. However,
% subfig.sty requires and automatically loads Axel Sommerfeldt's caption.sty
% which will override IEEEtran.cls' handling of captions and this will result
% in non-IEEE style figure/table captions. To prevent this problem, be sure
% and invoke subfig.sty's "caption=false" package option (available since
% subfig.sty version 1.3, 2005/06/28) as this is will preserve IEEEtran.cls
% handling of captions.
% Note that the Computer Society format requires a sans serif font rather
% than the serif font used in traditional IEEE formatting and thus the need
% to invoke different subfig.sty package options depending on whether
% compsoc mode has been enabled.
%
% The latest version and documentation of subfig.sty can be obtained at:
% http://www.ctan.org/pkg/subfig




% *** FLOAT PACKAGES ***
%
%\usepackage{fixltx2e}
% fixltx2e, the successor to the earlier fix2col.sty, was written by
% Frank Mittelbach and David Carlisle. This package corrects a few problems
% in the LaTeX2e kernel, the most notable of which is that in current
% LaTeX2e releases, the ordering of single and double column floats is not
% guaranteed to be preserved. Thus, an unpatched LaTeX2e can allow a
% single column figure to be placed prior to an earlier double column
% figure.
% Be aware that LaTeX2e kernels dated 2015 and later have fixltx2e.sty's
% corrections already built into the system in which case a warning will
% be issued if an attempt is made to load fixltx2e.sty as it is no longer
% needed.
% The latest version and documentation can be found at:
% http://www.ctan.org/pkg/fixltx2e


%\usepackage{stfloats}
% stfloats.sty was written by Sigitas Tolusis. This package gives LaTeX2e
% the ability to do double column floats at the bottom of the page as well
% as the top. (e.g., "\begin{figure*}[!b]" is not normally possible in
% LaTeX2e). It also provides a command:
%\fnbelowfloat
% to enable the placement of footnotes below bottom floats (the standard
% LaTeX2e kernel puts them above bottom floats). This is an invasive package
% which rewrites many portions of the LaTeX2e float routines. It may not work
% with other packages that modify the LaTeX2e float routines. The latest
% version and documentation can be obtained at:
% http://www.ctan.org/pkg/stfloats
% Do not use the stfloats baselinefloat ability as the IEEE does not allow
% \baselineskip to stretch. Authors submitting work to the IEEE should note
% that the IEEE rarely uses double column equations and that authors should try
% to avoid such use. Do not be tempted to use the cuted.sty or midfloat.sty
% packages (also by Sigitas Tolusis) as the IEEE does not format its papers in
% such ways.
% Do not attempt to use stfloats with fixltx2e as they are incompatible.
% Instead, use Morten Hogholm'a dblfloatfix which combines the features
% of both fixltx2e and stfloats:
%
% \usepackage{dblfloatfix}
% The latest version can be found at:
% http://www.ctan.org/pkg/dblfloatfix




% *** PDF, URL AND HYPERLINK PACKAGES ***
%
%\usepackage{url}
% url.sty was written by Donald Arseneau. It provides better support for
% handling and breaking URLs. url.sty is already installed on most LaTeX
% systems. The latest version and documentation can be obtained at:
% http://www.ctan.org/pkg/url
% Basically, \url{my_url_here}.




% *** Do not adjust lengths that control margins, column widths, etc. ***
% *** Do not use packages that alter fonts (such as pslatex).         ***
% There should be no need to do such things with IEEEtran.cls V1.6 and later.
% (Unless specifically asked to do so by the journal or conference you plan
% to submit to, of course. )


% correct bad hyphenation here
\hyphenation{op-tical net-works semi-conduc-tor}


\begin{document}
%
% paper title
% Titles are generally capitalized except for words such as a, an, and, as,
% at, but, by, for, in, nor, of, on, or, the, to and up, which are usually
% not capitalized unless they are the first or last word of the title.
% Linebreaks \\ can be used within to get better formatting as desired.
% Do not put math or special symbols in the title.
\title{Random Walk Algorithms and Application: \\ A Survey }


% author names and affiliations
% use a multiple column layout for up to three different
% affiliations
\author{\IEEEauthorblockN{Michael Shell}
\IEEEauthorblockA{School of Electrical and\\Computer Engineering\\
Georgia Institute of Technology\\
Atlanta, Georgia 30332--0250\\
Email: http://www.michaelshell.org/contact.html}
\and
\IEEEauthorblockN{Homer Simpson}
\IEEEauthorblockA{Twentieth Century Fox\\
Springfield, USA\\
Email: homer@thesimpsons.com}
\and
\IEEEauthorblockN{James Kirk\\ and Montgomery Scott}
\IEEEauthorblockA{Starfleet Academy\\
San Francisco, California 96678-2391\\
Telephone: (800) 555--1212\\
Fax: (888) 555--1212}}

% conference papers do not typically use \thanks and this command
% is locked out in conference mode. If really needed, such as for
% the acknowledgment of grants, issue a \IEEEoverridecommandlockouts
% after \documentclass

% for over three affiliations, or if they all won't fit within the width
% of the page (and note that there is less available width in this regard for
% compsoc conferences compared to traditional conferences), use this
% alternative format:
%
%\author{\IEEEauthorblockN{Michael Shell\IEEEauthorrefmark{1},
%Homer Simpson\IEEEauthorrefmark{2},
%James Kirk\IEEEauthorrefmark{3},
%Montgomery Scott\IEEEauthorrefmark{3} and
%Eldon Tyrell\IEEEauthorrefmark{4}}
%\IEEEauthorblockA{\IEEEauthorrefmark{1}School of Electrical and Computer Engineering\\
%Georgia Institute of Technology,
%Atlanta, Georgia 30332--0250\\ Email: see http://www.michaelshell.org/contact.html}
%\IEEEauthorblockA{\IEEEauthorrefmark{2}Twentieth Century Fox, Springfield, USA\\
%Email: homer@thesimpsons.com}
%\IEEEauthorblockA{\IEEEauthorrefmark{3}Starfleet Academy, San Francisco, California 96678-2391\\
%Telephone: (800) 555--1212, Fax: (888) 555--1212}
%\IEEEauthorblockA{\IEEEauthorrefmark{4}Tyrell Inc., 123 Replicant Street, Los Angeles, California 90210--4321}}




% use for special paper notices
%\IEEEspecialpapernotice{(Invited Paper)}




% make the title area
\maketitle

% As a general rule, do not put math, special symbols or citations
% in the abstract
\begin{abstract}
The abstract goes here.
\end{abstract}

% no keywords




% For peer review papers, you can put extra information on the cover
% page as needed:
% \ifCLASSOPTIONpeerreview
% \begin{center} \bfseries EDICS Category: 3-BBND \end{center}
% \fi
%
% For peerreview papers, this IEEEtran command inserts a page break and
% creates the second title. It will be ignored for other modes.
\IEEEpeerreviewmaketitle



\section{Introduction}
% no \IEEEPARstart
\iffalse
This demo file is intended to serve as a ``starter file''
for IEEE Computer Society conference papers produced under \LaTeX\ using
IEEEtran.cls version 1.8b and later.
\fi
% You must have at least 2 lines in the paragraph with the drop letter
% (should never be an issue)

Random walk and quantum walk have been used broadly in computer science community during the past years. They play an important role in computer vision, recommender system, social network etc.\par
Random walk has a very long history, which was first introduced by Pearson in 1905\cite{Pearson1905The}. A well-known application of random walk is page rank algorithm\cite{Page1998The}. And page rank become an import kind of random walk proximity measures. There are other researchers present other proximity measure based on random walk.\cite{Kleinberg1999Authoritative}\cite{Haveliwala2003Topic} [Reversible Markov Chains]. Based on these proximity measures, a lot of widely used algorithms came up. In the era of collaborative filtering, researchers introduced some algorithm based on the proximity measures. \cite{fouss2005a} \cite{brand2005a}. proximity measures also play an important role in recommending systems. \cite{woodruff2000enhancing} \cite{mcnee2002on} \cite{gori2006research}. Random walk also used in computer vision\cite{A Random Walks View of Spectral Segmentation} \cite{von luxburg2007a} \cite{1704833}\cite{qiu2005image} and semi-supervised learning\cite{zhu2003semi-supervised} \cite{szummer2002partially} \cite{azran2007the} \cite{tishby2001data}.

\par During the past few years, there many application of random walk on the analysis of social network. Many researchers have done excellent job on random walk. There are barely no complete reviews about random walk in computer science community. We would like to summarize their methods and results. Moreover, we can find out the challenges of random walk method.\par
The rest of this paper are organized as follows. In section 2, we will introduction some basic concepts of random walk. In section 3, we will introduce some important proximity measures in random walk. In section 4, we will introduce the application of random walk in computer science, including collaborative filtering, recommending system. In section 5, we will introduce the random walk from a quantum view the quantum walks. In section 6, we will discuss about the challenge of random walk in computer science.
%%%%%%%%%%%%%%%%%%%%%%%%%%%%%%%%%%%%%%%%%%%%%%%%%%%%5
\iffalse


\hfill mds

\hfill August 26, 2015

\subsection{Subsection Heading Here}
Subsection text here.


\subsubsection{Subsubsection Heading Here}
Subsubsection text here.


\fi
%%%%%%%%%%%%%%%%%%%%%%%%%%%%%%%%%%%%55555
% An example of a floating figure using the graphicx package.
% Note that \label must occur AFTER (or within) \caption.
% For figures, \caption should occur after the \includegraphics.
% Note that IEEEtran v1.7 and later has special internal code that
% is designed to preserve the operation of \label within \caption
% even when the captionsoff option is in effect. However, because
% of issues like this, it may be the safest practice to put all your
% \label just after \caption rather than within \caption{}.
%
% Reminder: the "draftcls" or "draftclsnofoot", not "draft", class
% option should be used if it is desired that the figures are to be
% displayed while in draft mode.
%
%\begin{figure}[!t]
%\centering
%\includegraphics[width=2.5in]{myfigure}
% where an .eps filename suffix will be assumed under latex,
% and a .pdf suffix will be assumed for pdflatex; or what has been declared
% via \DeclareGraphicsExtensions.
%\caption{Simulation results for the network.}
%\label{fig_sim}
%\end{figure}

% Note that the IEEE typically puts floats only at the top, even when this
% results in a large percentage of a column being occupied by floats.


% An example of a double column floating figure using two subfigures.
% (The subfig.sty package must be loaded for this to work.)
% The subfigure \label commands are set within each subfloat command,
% and the \label for the overall figure must come after \caption.
% \hfil is used as a separator to get equal spacing.
% Watch out that the combined width of all the subfigures on a
% line do not exceed the text width or a line break will occur.
%
%\begin{figure*}[!t]
%\centering
%\subfloat[Case I]{\includegraphics[width=2.5in]{box}%
%\label{fig_first_case}}
%\hfil
%\subfloat[Case II]{\includegraphics[width=2.5in]{box}%
%\label{fig_second_case}}
%\caption{Simulation results for the network.}
%\label{fig_sim}
%\end{figure*}
%
% Note that often IEEE papers with subfigures do not employ subfigure
% captions (using the optional argument to \subfloat[]), but instead will
% reference/describe all of them (a), (b), etc., within the main caption.
% Be aware that for subfig.sty to generate the (a), (b), etc., subfigure
% labels, the optional argument to \subfloat must be present. If a
% subcaption is not desired, just leave its contents blank,
% e.g., \subfloat[].


% An example of a floating table. Note that, for IEEE style tables, the
% \caption command should come BEFORE the table and, given that table
% captions serve much like titles, are usually capitalized except for words
% such as a, an, and, as, at, but, by, for, in, nor, of, on, or, the, to
% and up, which are usually not capitalized unless they are the first or
% last word of the caption. Table text will default to \footnotesize as
% the IEEE normally uses this smaller font for tables.
% The \label must come after \caption as always.
%
%\begin{table}[!t]
%% increase table row spacing, adjust to taste
%\renewcommand{\arraystretch}{1.3}
% if using array.sty, it might be a good idea to tweak the value of
% \extrarowheight as needed to properly center the text within the cells
%\caption{An Example of a Table}
%\label{table_example}
%\centering
%% Some packages, such as MDW tools, offer better commands for making tables
%% than the plain LaTeX2e tabular which is used here.
%\begin{tabular}{|c||c|}
%\hline
%One & Two\\
%\hline
%Three & Four\\
%\hline
%\end{tabular}
%\end{table}


% Note that the IEEE does not put floats in the very first column
% - or typically anywhere on the first page for that matter. Also,
% in-text middle ("here") positioning is typically not used, but it
% is allowed and encouraged for Computer Society conferences (but
% not Computer Society journals). Most IEEE journals/conferences use
% top floats exclusively.
% Note that, LaTeX2e, unlike IEEE journals/conferences, places
% footnotes above bottom floats. This can be corrected via the
% \fnbelowfloat command of the stfloats package.
\section{Brief Introduction of Random Walk}

\par Random walk is an important part of stochastic process. Stochastic process can be denoted as  . Xt is a random variable. Single step transition probability can be denoted as $\{ \xi_{t}, t=0,1,2,... \}$  . T steps transition probability are defined as follow.

A graph is denoted by $G=(V,E)$, where V denotes the vertex set and E denoted the edge set. The adjacency matrix is denoted by A, where the Aij means the weigh on edge i,j. The transition probability (single step) between node I and node j on the graph can be defined as follow.
$pij=Aij/ $ . We employ the diagonal matrix D to record   for each node. In that case we can define the transition matrix of the graph.
$Pij=Aij/Dii$
P denotes the transition matrix of the graph.
The Laplacian of G can be defined as follow.
$L = D-A$



\section{Proximity Measures}



\subsection{PageRank}

The most famous Proximity measure is the page rank. It was first proposed by Lary Page. [The PageRank Citation Ranking: Bringing Order to the Web] The purpose of this measure is to rank the webpage in World Wide Web. The network of webpage is considered as a graph where random walk happens. The graph is made up by vertexes and edges. The webpages are considered as the vertexes. if there is a webpage containing a hyperlink which pointing to anther webpage, then there should be a directed edge between these two vertexes. The direction of the edge is same as the web redirection. The most simple page rank can be describe by the mathematic equation.


        \begin{equation}\label{pagerank}
        R(u) = c \sum_{v\in B_{u}} \frac{R(v)}{N_{v}}
        \end{equation}


R(u) is the rank of web u. B(u) is the set of vertexes point to page u. out(u) is the set of pages u points to. N(v) is the number of vertexes in set out(u). The intuition behind this equation is that a page is important when it has more backlinks and more important these backlinks are, higher rank this page gets.
But this simple page rank can��t be implemented because the practical situation is much more complicated. A more reliable mathematic description of page rank is as follows.


		\begin{equation}\label{advpagerank}
        V = (1- \alpha)P^{T}V+ \frac{\alpha}{n} \textbf{1}
        \end{equation}				


Alpha is the probability that the rand walk restart in given steps. Alpha is crucial for this proximity measure. It make sure the random walk procedure is aperiodic and irreducible. In that case, the random walk in the web network can converge to a certain distribution. However, the calculation of page rank uses a power method. To improve the converge speed of page rank, Quadratic Extrapolation [Extrapolation Methods for Accelerating PageRank Computations] present a novel algorithm for Page rank computation. Quadratic Extrapolation accelerate the convergence of the power method. The main strategy in this algorithm is periodically reducing estimates of the non-principal eigenvectors.


\subsection{HITS}


Page Rank has nothing to do with user-supplied query. Therefore, Jon M. Kleinberg [Authoritative sources in a hyperlinked environment] came up with ��Hyperlink-Induce topic search�� which can filter the search result for a broad topic. The author propose that there are two kinds of useful web-pages for topic search: authorities and hubs. He also proposed a link-base model for the conferral of authority. There are millions pages relevant to a broad topic. Authorities are the most central pages for the broad topic, which provide good information for the broad topic. Whereas the hubs are those pages contains hyper-links redirecting to the authorities. Now, we can discuss that main procedure of how the author provide a broad topic search result for the user. Given a query, the author construct a focused subgraph of the www relevant to the broad topic. The subgraph contains a set of relevant pages rich in candidate authorities. Then the author present a algorithm to discover authorities over the subgraph. The author would like to construct a subgraph denoted by S which satisfy the following requirements:
1.	S is relatively small
2.	S is rich in relevant pages
3.	S contains most the strongest authorities
How to construct subgraph S is the first difficult in HITS. The author rendered a solution:
1.	collect the t highest ranked pages for the query as the root set R
2.	expanding R along the links that enter and leave it
Now that we get a qualified subgraph, we can apply the algorithm over it. The author would like to extract these authorizes from the subgraph. Hence the author use two scores to describe a vertex in the subgraph: authority score and hub score. The intuition of this idea is that A node is good hub if it points to many authorities; a node is a good authority if it is pointed to by many good hubs. In order to break this circulation, the author use a iterative method, which can be mathematically describe as follows.

\iffalse
\begin{equation}

a(i)  \leftarrow \sum_{j:j \in I(i)} h(j)
h(i)  \leftarrow \sum_{j:j \in O(j)} a(j)

\end{equation}
\fi


I(i) denotes that the set of pages point to page i.
O(i) denotes that the set of pages pointed to by page i.
a(i) is the authority score of page I
h(i) is the hub score of page i
During the iteration, authority score and hub score are normalized so theire squares sum to 1. The two equation could be rewritten by using the matrix. $A$ denote the un-weighted adjacency matrix of the subgraph, vector a is the authority scores and vector h is the hub scores.


From the above equations, we can know that the hub scores converges to the principal eigenvector of \(AAt\), meanwhile the authority scores converge to the principal eigenvector of \(AtA\).



\subsection{Personalized PageRank}

Pagerank is a very democratic [The PageRank Citation Ranking: Bringing Order to the Web] since the walker can jump to every vertex with the Probability of alpha. On the contrary, personalized PageRank concentrate on one vertex. The intuitive idea of personalized page rank is the random walker can jump to a certain vertex with the probability of alpha.
The mathematical description is the following equation. [RANDOM WALKS IN SOCIAL NETWORKS AND THEIR APPLICATIONS: A SURVEY]


                \begin{equation}
                \textbf{v} = (1- \alpha)P^{T} \textbf{v} + \alpha \textbf{r}
                \end{equation}


There are many link prediction problem using personalized page rank as a proximity measure.[ The Link-Prediction Problem for Social Networks] We will discuss these applications of personalized pagerank in the following sections.



\subsection{Hitting Time and Commute Time}
\par
\textbf{Hitting time}��hitting time [Reversible Markov Chains] can be considered as a weighted path length from I to j. The mathematic definition of hitting time is as follows:
%\begin{equation}
\begin{comment}
\[ h_{ij} =
\begin{cases}
1+\sum_{k} p_{ik}h_{kj} &\textbf{If}i \neq j,\\
0 &\textbf{If}i = j.
\end{cases} \]

%\end{equation}
\end{comment}


$Hij$ denotes the hitting time from node $i$ to node $j$. $P_{ik}$ denotes the transition probability from node I to node k. As we have mentioned before, on the undirected graph, transition probability matrix is symmetric. However, the hitting time matrix is not symmetric even on the undirected graph. Another important fact about hitting time was proved by LOVASZ [Random walk on graphs: a survey]: hitting time follows the triangle inequality.
The commute time from node I to node j is defined as :
\begin{equation}
  C_{ij} =h_{ij} +h_{ji}
\end{equation}		


In order to research the commute time on undirected graphs, Ashok K. Chandra et al gave an electrical network view. They compare commute time between two nodes on graph to resistance on electrical network. They gave us some intuition about commute time on undirected graphs:
1.	The smaller resistance can make the current go through easier on electrical networks, the smaller commute time can make random walker diffuse easier on undirected graphs.
2.	commute time should be robust to small perturbation since removing or adding a few resistances do not change much on an electrical network.



\subsection{}



\section{Applications of Random Walk}
\subsection{Computer Vision}
Many researchers solve computer vision problems by using random walk. One of the common techniques is characterizing shape of image by using random walk. Gorelick et al. [Shape Representation and Classification Using the Poisson Equation] compute many useful properties of a silhouette based on the notion of random walk. For every internal pixel in the contour, they compute a value reflecting the mean time required for a random walker beginning at the pixel to reach the boundary. Based on the computed values, they can extract many properties of the silhouette such as part structure, rough skeleton, local orientation, convex part and concave part.
Random walk is also utilized in image segmentation. Meila et al. [A Random Walks View of Spectral Segmentation] present an approach of image clustering and segmentation based on the view of random walk proximity measures. They also find that the spectral view of clustering and segmentation have a probabilistic foundation. They exploit the eigenvalue and eigenvector of walker��s transition matrix to cluster and segment image. Grady et al. [Random Walks for Image Segmentation ] propose a new algorithm for performing multilabel, interactive image segmentation. The interactive image segmentation means that the user has to label some pixels in the image manually. Given these labeled pixels, the algorithm can quickly determine the probability that a random walker starting from an unlabeled pixel will first reach the predefined pixels. Therefore, a good segmentation of that image arises from the labels of all the pixels. The predefined labels indicate that the regions of the image belong to several objects. The authors treat the image as a graph including nodes and edges. Nodes represents the pixels of the image, and edges represents the connection of two nodes where the value means the likelihood of a random walker going through that edge. The authors believe that this view of image segmentation has following advantage: there is no discretization errors or ambiguities. Because the authors use purely combinational operators that require no discretization.  The segmentation algorithm only requires solution to a sparse, symmetric, positive definite system of equation, hence the efficiency of this algorithm is guaranteed. Qiu et al. [Image Segmentation using Commute times] exploits the properties of the commute time to develop image segmentation method. They compute the commute time from the spectrum. By using the discrete Green��s function of graphs, they can analyze the cuts of the image from commute time. Qiu also use commute time to motion track [Robust Multi-body Motion Track using Commute Time Clustering]. The main purpose of using commute time as proximity measure is to alleviate the effect of noise on the shape interaction matrix. The noise on the shape interaction matrix result in the loss of block-diagonal structure and the difficulty of the assignment of elements to objects. Commute time is a more robust measure than row proximity matrix when facing the noise on the shape interaction matrix. The authors compute the commute time by using the Laplacian matrix. They also show us that how the ensemble the commute time, kernel of PCA (Principle Component Analysis), the Laplacian eigenmap and the diffusion map. To demonstrate the result of the robust method, they compare it with some other motion tracking algorithms on both synthetic and real world data. The function of commute time is to provide a proximity measure for clustering in the literature of Qin[Robust Multi-body Motion Track using Commute Time Clustering][ Image Segmentation using Commute times].



\subsection{Recommender System }

Some scholars use random walk to solve their own problem during doing research. They find that it is hard for them to find out useful literature recently published in their field. A researcher is supposed to be well aware of recent development of the field he is working on. So a paper recommendation system can help them find out potential helpful papers. This is meaningful and time saving. Because publications increase exponentially, selecting useful papers really a pain in the neck for the most researchers.
	A very simplified algorithm to solve this problem was presented by Woodruff et al [Enhancing a Digital Book with a Reading Recommender]. The author employ spreading activation and citation data to generate recommendations. The author uses documents read by the reader as input. Then recommend the most related literature to the reader in a digital book. This method only recommends a chapter or an article in a digital book.
A more applicable method for paper recommendation can be found in [On the Recommending of Citations for Research Papers]. The author exploit collaborative filtering for recommendation. The use the citation web as the graph to create ratings. The author investigated six algorithms by do experiments on the subset of ResearchIndex. The best algorithms can either provide relevant recommendations or novel recommendations, but none of them can do the both. And The use of citation web can affect the recommendations greatly.
	Also based on the citation graph, Gori et al exploit the idea of page rank algorithm to solve the paper recommendation problem by presenting the PaperRank algorithm [Research Paper Recommender Systems: A Random-Walk Based Approach]. The author��s view is that utilizing the model expressed by the citation graph can help us find out valuable papers to suggest to a user. The author considered that the PaperRank algorithm must have two properties: propagation and attenuation. With propagation, if a paper is relevant to good papers in bibliography of researcher��s work, then we can find out that this paper maybe a good suggestion for him. With attenuation, the positive influence of good papers decreases if we move futher and further away from good papers on the citation graph. PageRank algorithm has both properties. The author borrows its idea to solve paper recommendation problem. The essential of PaperRank algorithm is a random walk based score algorithm.
	Xia et al [Scientific Article Recommendation: Exploiting Common Author Relations and Historical Preferences] incorporates author relations and historical preferences for scientific article recommendation. The authors build a graph based on the information on common authors relation, and they employ the random walk with restart to generate a recommendation list. Compared with some baseline algorithms, the algorithm presented in the literature called CARE performs better in precision, recall and F1 score. Most studies of paper recommendation have the same algorithms for all the researchers no matter what the researcher��s situation is. But CARE method takes researchers�� own features into consideration. Hence the CARE method is more accurate than the baseline algorithms.



\subsection{collaborative filtering}

collaborative filtering is a method of making automatic predictions about the interests of a user by collecting preferences or taste information from many


users. The assumption of collaborative filtering is that the two people who have the same taste on one issue will have the same interest on the other issue.


Much literature has recorded methods of collaborative filtering, with successful demonstrations of Bayesian, nonparametric, linear methods etc.[ Gediminas
Adomavicius and Alexander Tuzhilin. Recommendation technologies: Survey of current methods and possible extensions. MISRC working paper 03-29,http://misrc.umn.edu/workingpapers/abstracts/0329.aspx,May 2003.] All these methods are essentially the same. They are all match the individual to others based on there choices, and use combination of their experiences to predict future choices. But Brand et al [A random walks perspective on maximizing satisfaction and profit] introduced a random walk view to collaborative filtering. The goal of Brand is to find out what products a customer wants to buy next, what product categories are preferred by specific demographic groups. They derived a weighted association graph from a relational database. These weighted association graph include consumers and their web browsing behavior, shopping behavior and entertainment choices etc. \textbf{\underline{a fig of collaborative filtering}} The author look at the expected behavior of random walk on the association graph. Based on the hitting time and commute time, the authors employ a novel measure of similarity��the cosine correlation between states. Compared with other methods of collaborative filtering, one of the biggest advantages of random walks view is that it can incorporate large amounts of contextual information. By using cross-validation, the author proved that the random walk view collaborative filtering is more predictive and robust to perturbations of edges on the association graph than other methods. The flow of random walk view is the heavy computing price. In that case the author employ approximation strategies to alleviate time complexity. Fouss et al [A novel way of computing similarities between nodes of a graph, with application to collaborative recommendation] also use random walk to movie collaborative recommendation. The author also considers relational database as a collection of element sets linked by their connection. The author exploits the graph structure of the relational database to compute dissimilarity measure between elements in sets. The dissimilarity is of course based on the hitting time and commute time. It was also the first time that hitting time and commute time measures was used in collaborative recommendation. For better understanding, the author gives us a specific example of the collaborative movie recommendation. If we get three elements, people, movie and movie category, and two relationships between people and movie and between movie and movie category, we have to do following things for movie recommendation.
(1) compute dissimilarity measure between people based on the movies that they have watched\\
(2) compute dissimilarity measure between people and movies for recommendation\\
(3) compute dissimilarity measure between people and categories to give a prefer category for each person.\\
As a conclusion, Fouss et al introduced a general procedure for computing similarity between elements of a relational database. These elements are not necessarily directly connected. The authors use movie recommendation as an example to show us that their method has better performance than shortest path method on recommendation. However, there are two shortcomings of the author��s method. For large databases, this method is time consuming and does not scale well. The other short coming is that the method is valid on a weighted, undirected graph.
Although the random walk view of collaborative filtering is useful and has good performance, it also faces several challenges. One of the biggest problem is the start-up problem presented by Resnick [An Open Architecture for Collaborative Filtering of Netnews]. All collaborative filtering systems are based on an existed database. If there isn��t an existed database, the system can��t be built up.



\subsection{Semi-supervised Learning}

Semi-supervised learning uses both labeled data and unlabeled data for training. The goal is to classify the unlabeled data when the labeled data is just a small fraction of the dataset.

Zhu et al. [Semi-supervised learning using Gaussian Fields and harmonic functions] present a new approach of semi-supervised learning based on the random walk. They do classification task in continuous state space rather than in the discrete label set. The intuition of the approach is that the data points should be labeled the same as their neighbors. And their neighbors are given by the random walk on graph. The author��s strategy is to employ a harmonic function f: V->R on graph G. The harmonic function has a constrain on the labeled data i:
$f(i) \equiv fl(i)y(i)$. The harmonic function, which provides a consistent probabilistic semantics, is the basis of this semi-supervised classification approach. Since the author do classification in the continuous state space, they have to turn the continuous state space into discrete label set. Instead of employing a simple threshold in terms of the interpretation of random walk, the authors incorporate the prior knowledge by using CMN (class mass normalization) procedure. The promising result has shown that the approach can improve the accuracy of classification by exploiting the structure of unlabeled data.

Szummer et al [Partially labeled classification with Markov random walks] the partially labeled data may be in the submanifold space, hence a measure of global similarity is needed for semi-supervised learning. In the meanwhile, the authors also hope the measure can incorporate the structure of manifold. Based on these consideration, they present a Markov random walk model to classify the data. The research of [Data clustering by Markovian relaxation and the information Botteneck Method], which shows how to turn the distance matrix into a Markov process, helps a lot with the construction of graph. In that case, the representation of dataset arises naturally. Given a partially labeled data set   in which L is much smaller than N, the author represent the data set as a graph where node k represents the data (xk,yk) or xk. For node k, P0|t(i|k) denotes the probability of the random walker from node I to node k after t steps. They classify node k with the label c when c maximizes the following formula.
  y=c.
P(y|i) is an unknown parameter, which can be estimated by two techniques: maximum likelihood with EM, and maximum margin subject to constraints. They discuss the two techniques in the paper and empirically show that the margin estimation has better performance. In a word, the authors provide a novel approach for semi-supervised learning task when the data sets with significant manifold structure. The parameter t in this approach is also important. T , denoting the number of transitions, determines the smoothness of random walk. However, the choice of t can be tricky and subjective. To overcome this little problem, Azran [the rendezvous algorithm: multiclass semi-supervised learning with markov random walks] presents the rendezvous algorithm. Just the same as the work of Szummer [Partially labeled classification with Markov random walks], the author represents the data points as nodes of a graph and employ the random walk view to do classification. The intuition of [Partially labeled classification with Markov random walks] is the labels�� propagation over the graph. But the rendezvous algorithm is different. The labeled points don��t propagate, but absorb the states of the random walk. The probability of each unlabeled data to be absorbed by different labeled points can be used to derive a distribution as the transition steps increase to infinity. Hence the rendezvous algorithm doesn��t bother to choose a good value of the parameter t. The author draws a conclusion that the location of labeled point in the data set is important as the size of labeled data set in terms of the experiments�� results.

\section{Quantum view of Random Walk}

Kempe et al[quantum random walks - an introductory overvew] presented us two kinds of quantum walks. They are discrete time quantum walk and continus time random walk.

\subsection{Discrete Time Quantum Walk}
The discrete time model first appeared in the work of Feynman[Quantum mechanics and

path integrals.] in 1966. In the field of quantum computation, Meyer rediscovered the discrete time model of quantum walk in [ From quantum cellular automata to quantum

lattice gases.][On the absence of homogeneous scalar unitary
cellular automata ]. we define a space $H=H_{p}\bigotimes  H_{c}$ for one dimensional quantum walk, . $H_{p}$ denotes Hilbert space.  For one dimensional Hilbert space, it can be represented as follows.

$H_{p}=\lbrace \vert i\rangle : i \in Z\rbrace$

${H_{c}}$  is spanned by two basic states  $\lbrace \vert\uparrow  \rangle,\vert \downarrow\rangle \rbrace$.  Operation $S$ defines the translation on space $H$.
\begin{equation}
S =\vert \uparrow\rangle\langle\uparrow\vert\otimes\sum_{i}{\vert i+1\rangle\langle i \vert +\vert\downarrow\rangle\langle \downarrow\vert \otimes\sum_{i}{\vert i-1\rangle}}\langle i \vert
\end{equation}
S can transform the basic state $\vert \uparrow \rangle \otimes \vert i \rangle $ to $ \vert \uparrow \rangle \otimes \vert i+1\rangle $ and $\vert \downarrow\rangle\otimes \vert i\rangle to \vert \downarrow\rangle\otimes\vert i-1\rangle$.

C is a unitrary transformation to rotate the spin in $H_{c}$.  A frequently used unitary transformation is called Hadamard coin H.  Here is an example of H.
\begin{equation}
\vert \uparrow \rangle \otimes \vert 0\rangle\xrightarrow{H} \frac{1}{\sqrt{2}} (\vert \uparrow\rangle + \vert\downarrow\rangle)\otimes\vert0\rangle
\end{equation}
The single quantum walk transformation can be defined as follows.
\begin{equation}
 U = S\cdot(C\otimes I)
\end{equation}
Here is a example of single step transformation.

\begin{equation}
\vert \uparrow \rangle \otimes \vert 0\rangle \xrightarrow{U} \frac{1}{\sqrt{2}}(\vert \uparrow\rangle\otimes\vert1\rangle+\vert\downarrow\rangle\otimes\vert-1\rangle)
\end{equation}

The T steps of tansformation can be represented by $U^{T}$.



\subsection{Continuous Time Quantum Walk}


The original purpose of  continuous time quantum walk is  to speed up many a algorithm using classic random walks. The concept of continuous time quantum walk was first presented by Farhi et al. in 1997.[Quantum computation and Decision tree] The authors exploit quantum walk in the decision tree algorithm instead of classic random walk to . Different from discrete time quantum walk, continuous time quantum walk don't need a coin space $Hc$, taking place entirely in the Hilbert space $Hp$.[quantum random walks - an introductory overview] . The idea of continuous time quantum walk is from continuous random walk.  The continuous time random walk can be defined as
\begin{equation}
P(t) = exp(-Ht)P(0)
\end{equation}. Similarly, the unitary time evolution operator of continuous time quantum walk is

\begin{equation}
\hat{U}(t)=exp(-i\hat{H}t)
\end{equation}



\subsection{Properties of Quantum walk}


\subsection{Algorithms based on Quantum walk}
The first quantum walk algorithm 


\section{Conclusion}
The conclusion goes here.


\section{open issues}




% conference papers do not normally have an appendix



% use section* for acknowledgment
\ifCLASSOPTIONcompsoc
  % The Computer Society usually uses the plural form
  \section*{Acknowledgments}
\else
  % regular IEEE prefers the singular form
  \section*{Acknowledgment}
\fi


The authors would like to thank...





% trigger a \newpage just before the given reference
% number - used to balance the columns on the last page
% adjust value as needed - may need to be readjusted if
% the document is modified later
%\IEEEtriggeratref{8}
% The "triggered" command can be changed if desired:
%\IEEEtriggercmd{\enlargethispage{-5in}}

% references section

% can use a bibliography generated by BibTeX as a .bbl file
% BibTeX documentation can be easily obtained at:
% http://mirror.ctan.org/biblio/bibtex/contrib/doc/
% The IEEEtran BibTeX style support page is at:
% http://www.michaelshell.org/tex/ieeetran/bibtex/
%\bibliographystyle{IEEEtran}
% argument is your BibTeX string definitions and bibliography database(s)
%\bibliography{IEEEabrv,../bib/paper}
%
% <OR> manually copy in the resultant .bbl file
% set second argument of \begin to the number of references
% (used to reserve space for the reference number labels box)
%\begin{thebibliography}{}
\iffalse
\bibitem{IEEEhowto:kopka}
H.~Kopka and P.~W. Daly, \emph{A Guide to \LaTeX}, 3rd~ed.\hskip 1em plus
  0.5em minus 0.4em\relax Harlow, England: Addison-Wesley, 1999.
\fi
%\end{thebibliography}
\bibliography{refer}
\bibliographystyle{IEEEtran}




% that's all folks
\end{document}


